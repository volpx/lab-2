\documentclass{article}
\usepackage[a4paper,]{geometry}
\usepackage{lmodern}
%\usepackage[export]{adjustbox}
\usepackage[utf8]{inputenc}
\usepackage[T1]{fontenc}
\usepackage{graphicx}
%\usepackage{titlepic}
\usepackage{mathtools}
%\usepackage{amsmath}
%\usepackage{amssymb}
%\usepackage{amsfonts}
\usepackage{wasysym}
%\usepackage{accents}
%\usepackage{esvect}
%\usepackage{subcaption}
\usepackage{multicol}
\usepackage{hyperref}
%\usepackage{enumitem}
\usepackage[makeroom]{cancel}
\usepackage{siunitx}
\usepackage{float}
\usepackage{lipsum}
\usepackage{textcomp}
\usepackage{circuitikz}
\usepackage{placeins}

\sisetup{separate-uncertainty=true}
\linespread{1.3}

% my commands
\newcommand{\E}[1]{\, \mathrm{e}{#1} \, }
\newcommand{\de}{\mathrm{d}}
\newcommand{\pars}{\mathbin{\!/\mkern-5mu/\!}}
\newcommand{\equalexpl}[1]{
	\underset{\substack{\uparrow\\\mathrlap{\text{#1}}}}{=}}

% circuits stuff
\tikzset{
  declare function={% in case of CVS which switches the arguments of atan2
    atan3(\a,\b)=ifthenelse(atan2(0,1)==90, atan2(\a,\b), atan2(\b,\a));},
  kinky cross radius/.initial=+.125cm,
  @kinky cross/.initial=+, kinky crosses/.is choice,
  kinky crosses/left/.style={@kinky cross=-},kinky crosses/right/.style={@kinky cross=+},
  kinky cross/.style args={(#1)--(#2)}{
    to path={
      let \p{@kc@}=($(\tikztotarget)-(\tikztostart)$),
          \n{@kc@}={atan3(\p{@kc@})+180} in
      -- ($(intersection of \tikztostart--{\tikztotarget} and #1--#2)!%
             \pgfkeysvalueof{/tikz/kinky cross radius}!(\tikztostart)$)
      arc [ radius     =\pgfkeysvalueof{/tikz/kinky cross radius},
            start angle=\n{@kc@},
            delta angle=\pgfkeysvalueof{/tikz/@kinky cross}180 ]
      -- (\tikztotarget)}}}



\title{Diodi}
%\subtitle{Caratteristiche e ponte di Graetz}
\author{Filippo Dal Farra \and Matteo Zandegiacomo Orsolina}
\date{10 Dicembre 2018}

\begin{document}

\maketitle

\newpage

\section{Introduzione}

Questa relazione riassume gli esiti di tre esperienze tutte finalizzate all'adempimento dell'analisi di un circuito RLC svolto durante l'ultima seduta. Per fare ci\'o si \'e sfruttato l'uso di un oscilloscopio il quale, da come si vedr\'a, modifica inevitabilmente le risposte dei circuiti in analisi. A causa di questo è risultato poi necessario derivare i valori delle capacità e delle resistenze parassite che si trovano all'interno dell'oscilloscopio per poi effettuare l'analisi tenendo conto di esse. \\

Successivamente abbiamo studiato la risposta in frequenza del circuito RC nelle diverse configurazioni di filtro passa alto e filtro passa basso, sempre variando le frequenze di taglio. Così facendo si \'e riusciti ad ottenere i diagrammi di Bode corrispondenti che associati alle relative funzioni di trasferimento teoriche forniscono i valori dei componenti usati nei vari casi. Inoltre queste misure sono poi state messe a confronto con i risultati ottenuti precedentemente nella carica e scarica del condensatore, per osservare se risultavano tra loro compatibili. \\

Infine abbiamo considerato anche un induttore, consistente in una bobina realizzata precedentemente. Inizialmente abbiamo studiato un circuito RL per trovate il valore dell'induttanza di questo elemento attraverso diversi cicli di carica e scarica del campo magnetico da esso generato. In seguito è stata posta all'interno di un circuito RLC composto dalla bobina e dal condensatore usato precedentemente di cui si sapeva a questo punto il valore. Ciò ci ha permesso di studiare il funzionamento del circuito passa banda e trovare la sua frequenza di risonanza. \\

Il codice utilizzato in questa esperienza \'e raggiungibile al \href{https://gitlab.com/volpx/lab-2/tree/master/Exp234}{<link>}.

\newpage

\section{Materiali e strumenti}

\begin{itemize}
  \item diodi 1N4007 oggetto di analisi
  
\end{itemize}

\newpage

\section{Procedure di misura}

Innanzitutto è stato necessario montare il circuito RC per osservare la scarica del condensatore come suggerito dalla scheda fornita \ref{fig:RC_LP}. Ad esso andava poi collegato l'oscilloscopio tramite i cavi coassiali all'ingresso ed all'uscita del circuito. Abbiamo impostato il generatore di funzioni in modo che  producesse lo scalino da $\Delta V$ a $0\ V$ , e in questa configurazione \'e stato osservato l'andamento in uscita dell'onda nella fase di scarica. Sono stati utilizzati 5 diversi valori di resistenza con lo stesso condensatore in modo da avere tempi caratteristici $\tau$ differenti e per ognuno sono state salvate 6 forme d'onda con l'oscilloscopio impostato a 16 averagings. Ciò ha fornito un numero sufficiente di dati per poi poter trovare i parametri relativi ad eventuali componenti parassiti, dato che in seguito la procedura \'e stata ripetuta senza il condensatore in esame, supplito dalla capacit\'a intrinseca di cavi e ADC dell'oscilloscopio. Tutti i valori di resistenza utilizzati sono stati misurati con il DMM. \\

Successivamente lo stesso circuito \'e stato studiato come filtro, analizzando la sua risposta ad un ingresso sinusoidale in funzione della frequenza applicata. Sono stati scelti valori di resistenza che ci dessero valori della frequenza di taglio specifici. Il generatore di funzioni \'e stato impostato a $V_{PP} = 2\div 4 V$ con frequenze distribuite esponenzialmente per una decade e \textonehalf \ prima e dopo la frequenza di taglio. Quindi sono stati da noi trascritti i valori di ampiezza in entrata ed uscita dal circuito assieme allo sfasamento tra i due forniti dall'oscilloscopio tramite le sue funzioni di misura in modo da poter creare il diagramma di Bode. Infine abbiamo applicato una resistenza in modo da caricare il circuito ed in queste condizioni \'e stata misurata l'impedenza in uscita del filtro con due diverse frequenze. La procedura è stato ripetuta con la configurazione passa alto \ref{fig:RC_HP1}, per un unico valore di resistenza. \\


In seguito si è considerata una bobina costruita precedentemente e con essa è stato costruito un circuito RL. Per fare ciò la bobina è stata inserita all'interno di un core di materiale ferromagnetico, il quale aveva il ruolo di amplificare l'induttanza da esso prodotta. Si è innanzitutto ancora usata un'onda quadra dal generatore di funzioni, con diverse configurazioni di resistenze in modo da studiarne la scarica. Si è poi montato un circuito RLC passa banda, sfruttando come induttore ancora la bobina e come condensatore l'elemento di circuito usato precedentemente. Si è impostato il generatore di funzioni in modo da realizzare un'onda sinusoidale applicata a due diverse configurazioni di valori di resistenza. Si è studiato ciò modificando i valori della frequenza, infittendo la risoluzione in concomitanza della frequenza di risonanza del circuito. \\

\newpage

%%%%%%%%%%%%% ANAL-isis %%%%%%%%%%%%%%%%%%%
\section{Analisi dei dati}

Verranno ora analizzati i dati ottenuti per ogni configurazione.
\subsection{Diodi} 
L'esperienza è cominciata con lo studio del comportamento di alcuni diodi, sfruttando il circuito \ref{fig:cD}.

\begin{figure}[h]
    \begin{center}
    \begin{circuitikz} []
    \draw
        (0,2) to [battery] (0,0)
        (0,0) to (3,0)
        (0,2) to [ammeter] (2,2) 
        (2,2) to [Do] (2,0)
        (2,2) to (3,2)
        (3,2) to [open, *-*] (3,0);
    \end{circuitikz}
    \caption{Circuito usato per l'analisi dei diodi}
    \label{fig:cD}
    \end{center}
\end{figure}

Per ognuno dei quattro diodi a nostra disposizione abbiamo preso le misure di come varia $i$ in funzione di $\Delta V$. Qui esponiamo solo i dati di uno dei diodi. I valori che otteniamo rappresentano una curva che assomiglia ad un'esponenziale, per cui proviamo a linearizzare il grafico. Ciò che otteniamo sono i due grafici riprodotti in figura \ref{fig:gDe} e \ref{fig:gDl}. 

\begin{figure}[h]
    \centering
    \begin{minipage}{0.49\textwidth}
        \centering
        \includegraphics[width=\textwidth]{fig1D.pdf} 
        \caption{Grafico esponenziale di $i$ in funzione di $\Delta V$}
        \label{fig:gDe}
    \end{minipage}\hfill
    \begin{minipage}{0.49\textwidth}
        \centering
        \includegraphics[width=\textwidth]{fig2D.pdf} 
        \caption{Grafico linearizzato di $\ln{i}$ in funzione di $\Delta V$ con relativo modello dalla regressione lineare}
        \label{fig:gDl}
    \end{minipage}
    
\end{figure}
    
Proviamo quindi a cercare i parametri della retta ottenuta linearizzando il grafico attraverso una regressione lineare. Assumiamo che l'incertezza di $\Delta V$, cioè dovuta al DMM sia nulla. Mentre per trovare l'incertezza su $i$, $\sigma[i]$, assumiamo che all'interno dell'intervallo di incertezza massimo ci sia equiprobabilità tra tutti i possibili valori, per cui non possiamo sapere niente. Ovviamente quindi a seconda del fondo scala preso in considerazione varia l'incertezza sulla misura, che sarà cioè maggiore per i valori aventi una corrente più elevata.
Per ottenere i valori di intercetta e pendenza prendiamo il grafico di $\Delta V$ in funzione di $i$, e a partire da esso quello di $\Delta V$ in funzione di $\ln{i}$. Attraverso il metodo dei minimi quadrati troviamo ora i valori di intercetta e pendenza di questo grafico, che chiameremo $A_p$ e $B_p$ rispettivamente. Ora non ci resta che invertire gli assi e propagare le incertezze sui nuovi parametri della retta e otteniamo il modello che descrive l'andamento, con dei valori $A$ e $B$ definitivi. \\

\begin{gather}
    \sigma[V]=0 \quad \sigma[i]= \frac{I_{FS}}{50 \sqrt{12}} \quad \sigma[\ln{i}] = \frac{\sigma[i]}{i}
    \\
    B = \frac{1}{B_p} = 24 \quad A = - A_p B = -21.8
    \\
    \sigma[B] = \frac{\sigma[B_p]}{B_p^2} = 1 \quad \sigma[A] = \sqrt{(B \sigma[A_p])^2 + (A_p \sigma[B])^2} = 0.9
    \label{eq:regr1}
\end{gather}

A parte qualche punto che si discosta più degli altri osserviamo  che il modello descrive bene i dati ottenuti. Infatti tutti i punti rientrano entro al massimo un $3\sigma$ dal modello, tranne gli ultimi due in cui potrebbero influire effetti di surriscaldamento del diodo o altro. Perciò i valori di $i$ in funzione di $\Delta V$ seguono effettivamente un andamento esponenziale. \\
Un risultato analogo si è ottenuto anche per gli altri tre diodi da noi studiati. \\

Consideriamo ora il diodo Zener, e studiamo il suo comportamento in modo analogo a quanto fatto con i diodi rettificatori. In particolare il circuito da noi considerato è quello rappresentato in figura \ref{fig:cZ}

\begin{figure}[h]
    \begin{center}
    \begin{circuitikz} []
    \draw
        (0,2) to [battery] (0,0)
        (0,0) to (3,0)
        (0,2) to [ammeter] (2,2) 
        (2,0) to [zzDo] (2,2)
        (2,2) to (3,2)
        (3,2) to [open, *-*] (3,0);
    \end{circuitikz}
    \caption{Circuito usato per l'analisi dei diodi Zener}
    \label{fig:cZ}
    \end{center}
\end{figure}

Anche in questo caso studiamo $i$ in funzione di $\Delta V$ e ancora usiamo le stesse modalità per trovare i valori a cui siamo interessati. I calcoli sono li stessi di prima, perciò qui forniamo solo i risultati numerici.

\begin{gather}
    B = 4.30 \quad A = -25.3
    \\
    \sigma[B] = 0.04 \quad \sigma[A] = 0.3
    %\label{eq:regr2}
\end{gather}

Andiamo quindi a studiare i grafici che otteniamo e osserviamo se il modello rispetto la previsione teorica. I grafici che otteniamo sono \ref{fig:gZe} e \ref{fig:gZl}

\begin{figure}
    \centering
    \begin{minipage}{0.49\textwidth}
        \centering
        \includegraphics[width=\textwidth]{fig1Z.pdf} 
        \caption{Grafico esponenziale di $i$ in funzione di $\Delta V$}
        \label{fig:gZe}
    \end{minipage}\hfill
    \begin{minipage}{0.49\textwidth}
        \centering
        \includegraphics[width=\textwidth]{fig2Z.pdf} 
        \caption{Grafico linearizzato di $\ln{i}$ in funzione di $\Delta V$ con relativo modello}
        \label{fig:gZl}
    \end{minipage}
\end{figure}

Si nota tuttavia che nel grafico linearizzato i valori non risultano compatibili. In particolare sono rappresentate due semirette, spezzate in concomitanza del punto numero 9 (in ordine crescente di corrente e tensione) della serie. Si suppone che in quel particolare punto si entri in un differente regime di funzionamento del diodo che ha provocato una variazione dei parametri. Si cercano quindi due nuovi modelli in grado di descrivere l'andamento dei punti prima e dopo questo punto. I nuovi parametri che si ottengono sono A1 e B1 per il primo tratto e A2 e B2 per il secondo. Li calcoliamo sempre allo stesso modo, per cui nuovamente inseriamo qui solo i risultati numerici finali.

\begin{gather}
    B_1 = 2.91 \quad A_1 = -18.9 \quad \sigma[B_1] = 0.08 \quad \sigma[A_1] = 0.5
    \\
    B_2 = 5.8 \quad A_2 = -32 \quad \sigma[B_2] = 0.2 \quad \sigma[A_2] = 1
    \\
    \label{eq:regr2}
\end{gather}

\begin{figure} [h]
    \centering
    \includegraphics[width=0.7\textwidth]{fig3Z.pdf} 
    \caption{Grafico di $\ln{i}$ in funzione di $\Delta V$ con i due modelli}
    \label{fig:gZc}
\end{figure}

Inseriamo quindi i nuovi valori all'interno di un grafico e osserviamo che questi due modelli descrivono alla perfezione i dati ottenuti nelle rispettive metà. Ciò avvalora dunque la tesi che effettuando la misura del punto 9 ci sia stata una modifica del circuito. Il grafico che otteniamo è fig. \ref{fig:gZc}. \\ 


Passiamo ora allo studio della resistenza dinamica. Essa rappresenta una delle caratteristiche tipiche dei diodi Zener che si modifica a seconda della corrente che scorre attraverso di esso. Otteniamo una curva che ha la forma di un'esponenziale. Per esserne certi consideriamo il logaritmo sia della corrente che della resistenza dinamica del nostro diodo e attraverso la regressione lineare osserviamo se ciò che otteniamo rappresenta o meno un andamento lineare. I valori che otteniamo sono i seguenti. Per quanto riguarda i risultati sulla regressione lineare non riscriviamo le formule, in quanto sempre le stesse. \\

\begin{gather}
    r_d = \frac{\partial V}{\partial i} \quad \sigma[r_d] = \frac{V \sigma[i]}{i^2} \quad \sigma[\ln{r_d}] = \frac{\sigma[r_d]}{r_d}
    \\
    A = -2.39 \quad B = -1.249 \quad \sigma[A] = 0.01 \quad \sigma[B] = 0.002
    \\
    \label{eq:regr3}
\end{gather}

Proviamo ad inserire quindi i risultati ottenuti nel grafico \ref{fig:gZr}, e troviamo quindi che l'andamento è rappresentato accettabilmente dalla retta calcolata avente i parametri sopra indicati. \\

\begin{figure} [h]
    \centering
    \includegraphics[width=\textwidth]{fig4Z.pdf} 
    \caption{Grafico di $\ln{r_d}$ in funzione di $\ln{i}$ con il relativo modello ottenuto tramite regressione lineare.}
    \label{fig:gZr}
\end{figure}

Osserviamo inoltre che gli intervalli di incertezza di tutti i punti rientrano in almeno 3$\sigma$ dalla retta calcolata, per cui possiamo dire che effettivamente l'andamento da noi ottenuto risulta corretto. 
\subsection{Ponte di Graetz} 

Verr\'a prima preso in esame il ponte di graetz come rettificatore a doppia semionda filtrato ma senza stabilizzatore a diodo zener in fig \ref{fig:bridge}.

\begin{figure}[h]
\begin{center}
	\begin{circuitikz} []
	\draw
	(0,3.05) node[transformer core](T){}
	(T.B1) to ++(0,.45) to [short,l=$\AC 7.5\, \si{\volt}_{\textrm{rms}}$] (3,3.5) to (3,3)
	(T.B2) to ++ (0,-0.45) to (3,0.5) to (3,1)
	(T.A1) to [short,l=$\AC 220\, \si{\volt}_{\textrm{rms}}$] ++ (-.5,0)
	(T.A2) to ++ (-.5,0)
	%(T.B2)+(1,0) to [open,-, l=$\AC 7.5\, \si{\volt}$] (T.B1)
	
	(2,2) to [Do] (3,3) to [Do] (4,2)
	(2,2) to [Do] (3,1) to [Do] (4,2)
	(2,2) to  (2,0.55)
	(2,0.45) to (2,0) to (5,0)
	(4,0) node[ground] {}
	(4,2) to (5,2) to [eC,l=C] (5,0)
	(5,2) to (6,2) to [R, l=$R_l$] (6,0) to (5,0) 
	(6,2) to (7,2) to [open, -*] (7,2) node[right] {$V_{out}$}
	
	;
	\end{circuitikz}
\end{center}
\caption{Ponte di Graetz con filtro capacitivo.}
\label{fig:bridge}
\end{figure}

Il condensatore elettrolitico scelto $C=220\ \si{\micro\farad}$ ha lo scopo di stabilizzare la doppia semionda in uscita dal ponte risultando nell'onda in fig \ref{fig:brid_cfilter}.

\begin{figure}[h]
\centering
% TODO: substitute the image asking to someone for data
\includegraphics[width=\textwidth]{cripplefilter.pdf}
\caption{In blu la forma d'onda in uscita con il condensatore ($Rl=200\ \si{\ohm}$) ed in arancione se esso non ci fosse (quest'ultima forma d'onda \'e stata generata numericamente non potendole osservare contemporaneamente)}
\label{fig:brid_cfilter}
\end{figure}

Si pu\'o procedere a calcolare ora la massima ddp che otterrei in funzione della resistenza di carico applicata e la si confronta con i valori misurati in fig. \ref{fig:vmaxcconf}. La corrente sul carico $i_l$ \'e stata calcolata dalla $V_{out}$ misurata.

\begin{gather}
	i_{l\, max} = \frac{V_{out\, max}}{R_l} \\
	V_{out\, max} = V_{in\, max} - 2 V_d(i_{l\, max}) 
\end{gather}

\begin{figure}[h]
\centering
\includegraphics[width=\textwidth]{fig1.pdf}
\caption{$Vin$ misurata prima del ponte a diodi e $Vout$ misurata ai capi del carico}
\label{fig:vmaxcconf}
\end{figure}

Come si pu\'o vedere la presenza dei diodi provoca effettivamente un calo di tensione come ci si aspetta e anche l'andamento in funzione del carico \'e quello previsto.

Il ripple invece e mostrato in fig. \ref{fig:vppcconf}.

\begin{gather}
	V_{out\, pp} = \frac{i_{l\, max}}{2 f C}
\end{gather}

\begin{figure}[h]
\centering
\includegraphics[width=\textwidth]{fig2.pdf}
\caption{Ripple su $Vout$ misurata ai capi del carico}
\label{fig:vppcconf}
\end{figure}

Il ripple sul condensatore segue qualitativamente l'andamento previsto e risulta del tutto compatibile per un range di carichi da $R_l=2\ \si{\kilo\ohm}$ a $R_l=30\ \si{\kilo\ohm}$.

\FloatBarrier
\newpage

Si aggiunge quindi ora il diodo zener per stabilizzare ulteriormente l'uscita secondo il circuito in figura \ref{fig:bridgez}.

\begin{figure}[h]
\begin{center}
	\begin{circuitikz} []
	\draw
	(0,3.05) node[transformer core](T){}
	(T.B1) to ++(0,.45) to [short,l=$\AC 7.5\, \si{\volt}_{\textrm{rms}}$] (3,3.5) to (3,3)
	(T.B2) to ++ (0,-0.45) to (3,0.5) to (3,1)
	(T.A1) to [short,l=$\AC 220\, \si{\volt}_{\textrm{rms}}$] ++ (-.5,0)
	(T.A2) to ++ (-.5,0)
	%(T.B2)+(1,0) to [open,-, l=$\AC 7.5\, \si{\volt}$] (T.B1)
	
	(2,2) to [Do] (3,3) to [Do] (4,2)
	(2,2) to [Do] (3,1) to [Do] (4,2)
	(2,2) to  (2,0.55)
	(2,0.45) to (2,0) to (5,0)
	(4,0) node[ground] {}
	(4,2) to (5,2) to [eC,l=C] (5,0)
	(5,2) to [R,l=R,i>_=$i_R$] (7,2) to [zzDo,invert,i>_=$i_z$] (7,0)
	(7,2) to (8,2) to [R, l=$R_l$,i>_=$i_l$] (8,0) to (5,0) 
	(8,2) to (9,2) to [open, -*] (9,2) node[right] {$V_{out}$}
	(5,2) to (5,3) to (9,3) to [open, -*] (9,3) node[right] {$V_c$}
	
	;
	\end{circuitikz}
\end{center}
\caption{Ponte di Graetz con filtro capacitivo e diodo zener.}
\label{fig:bridgez}
\end{figure}

In questa configurazione il diodo svolge la funzione di diminuire il ripple (fig. \ref{fig:ripplecomp}) che si era visto con solamente il condensatore al costo di dissipare potenza nella resistenza $R$ e nello zener.

\begin{figure}[h]
\centering
\includegraphics[width=\textwidth]{ripplecomp.pdf}
\caption{Confronto ripple con $R_l = 200 \ \si{\ohm}$.}
\label{fig:ripplecomp}
\end{figure}

Chiamata $i_z(V_z)$ la caratteristica tensione-corrente dello zener ed utilizzando la $V_{out}$ misurata per il calcolo di $i_l$ e $i_z$ si pu\'o definire:

\begin{gather}
	i_{l\, max} = \frac{V_{out\, max}}{R_l} \\
	i_{z\, max} = i_z(V_{out\, max}) \\
	i_{R\, max} = i_{z\, max} + i_{out\, max} \\
	V_{out\, max} = V_{c\, max} - i_{R\, max} R
\end{gather}

\begin{figure}[h]
\centering
\includegraphics[width=\textwidth]{fig4.pdf}
\caption{$V_{out}$ in funzione del carico}
\label{fig:vdiod}
\end{figure}

Come si pu\'o vedere la presenza dello stadio ulteriore di filtraggio diminuisce l'output del circuito.

Considerata $r_z(i_z)$ la resistenza dinamica del diodo, il ripple viene stimato in questo modo:

\begin{gather}
	V_{c\, pp} = \frac{i_{R\, max}}{2fC} \\
	V_{out\, pp} = V_{c\, pp} \frac{r_z}{r_z+R(1+\frac{r_z}{R_l})}
\end{gather}

\begin{figure}[h]
\centering
\includegraphics[width=\textwidth]{fig5.pdf}
\caption{Ripple in funzione del carico}
\label{fig:vdiodpp}
\end{figure}











\end{document}
